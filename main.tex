%% For double-blind review submission, w/o CCS and ACM Reference (max submission space)
\documentclass[sigplan,10pt,review]{acmart}\settopmatter{printfolios=true,printccs=false,printacmref=false}
%% For double-blind review submission, w/ CCS and ACM Reference
%\documentclass[sigplan,10pt,review,anonymous]{acmart}\settopmatter{printfolios=true}
%% For single-blind review submission, w/o CCS and ACM Reference (max submission space)
%\documentclass[sigplan,10pt,review]{acmart}\settopmatter{printfolios=true,printccs=false,printacmref=false}
%% For single-blind review submission, w/ CCS and ACM Reference
%\documentclass[sigplan,10pt,review]{acmart}\settopmatter{printfolios=true}
%% For final camera-ready submission, w/ required CCS and ACM Reference
%\documentclass[sigplan,10pt]{acmart}\settopmatter{}


%% Conference information
%% Supplied to authors by publisher for camera-ready submission;
%% use defaults for review submission.
\acmConference[PL'17]{ACM SIGPLAN Conference on Programming Languages}{January 01--03, 2017}{New York, NY, USA}
\acmYear{2017}
\acmISBN{} % \acmISBN{978-x-xxxx-xxxx-x/YY/MM}
\acmDOI{} % \acmDOI{10.1145/nnnnnnn.nnnnnnn}
\startPage{1}

%% Copyright information
%% Supplied to authors (based on authors' rights management selection;
%% see authors.acm.org) by publisher for camera-ready submission;
%% use 'none' for review submission.
\setcopyright{none}
%\setcopyright{acmcopyright}
%\setcopyright{acmlicensed}
%\setcopyright{rightsretained}
%\copyrightyear{2017}           %% If different from \acmYear

%% Bibliography style
\bibliographystyle{ACM-Reference-Format}
%% Citation style
%\citestyle{acmauthoryear}  %% For author/year citations
%\citestyle{acmnumeric}     %% For numeric citations
%\setcitestyle{nosort}      %% With 'acmnumeric', to disable automatic
                            %% sorting of references within a single citation;
                            %% e.g., \cite{Smith99,Carpenter05,Baker12}
                            %% rendered as [14,5,2] rather than [2,5,14].
%\setcitesyle{nocompress}   %% With 'acmnumeric', to disable automatic
                            %% compression of sequential references within a
                            %% single citation;
                            %% e.g., \cite{Baker12,Baker14,Baker16}
                            %% rendered as [2,3,4] rather than [2-4].


%%%%%%%%%%%%%%%%%%%%%%%%%%%%%%%%%%%%%%%%%%%%%%%%%%%%%%%%%%%%%%%%%%%%%%
%% Note: Authors migrating a paper from traditional SIGPLAN
%% proceedings format to PACMPL format must update the
%% '\documentclass' and topmatter commands above; see
%% 'acmart-pacmpl-template.tex'.
%%%%%%%%%%%%%%%%%%%%%%%%%%%%%%%%%%%%%%%%%%%%%%%%%%%%%%%%%%%%%%%%%%%%%%


%% Some recommended packages.
\usepackage{booktabs}   %% For formal tables:
                        %% http://ctan.org/pkg/booktabs
\usepackage{subcaption} %% For complex figures with subfigures/subcaptions
                        %% http://ctan.org/pkg/subcaption


\newcommand{\atl}{ATLAS }
\newcommand{\gem}{GEMM }

\begin{document}

%% Title information
\title[Short Title]{Capri-Based ATLAS: A Faster and Better Linear Algebra Auto-tunning Library}         %% [Short Title] is optional;
                                        %% when present, will be used in
                                        %% header instead of Full Title.
\titlenote{with title note}             %% \titlenote is optional;
                                        %% can be repeated if necessary;
                                        %% contents suppressed with 'anonymous'


%% Author information
%% Contents and number of authors suppressed with 'anonymous'.
%% Each author should be introduced by \author, followed by
%% \authornote (optional), \orcid (optional), \affiliation, and
%% \email.
%% An author may have multiple affiliations and/or emails; repeat the
%% appropriate command.
%% Many elements are not rendered, but should be provided for metadata
%% extraction tools.

%% Author with single affiliation.
\author{Yan Pei}
\affiliation{
  \department{Department of Computer Science}  %% \department is recommended
  \institution{University of Texas at Austin}  %% \institution is required
}
\email{ypei@cs.utexas.edu}          %% \email is recommended

%% Author with two affiliations and emails.
\author{Jiayuan He}
\affiliation{
  \department{Department of Computer Science}  %% \department is recommended
  \institution{University of Texas at Austin}  %% \institution is required
}
\email{hejy@cs.utexas.edu.com}         %% \email is recommended

%% Abstract
%% Note: \begin{abstract}...\end{abstract} environment must come
%% before \maketitle command
\begin{abstract}

Auto-tuning is a general approach to find the optimal optimization parameters such as tile sizes in matrix multiplication.
There are two main approaches in auto-tuning: empirical search and model-based.
Empirical search is universal but slow, and some heuristic to reduce search space may prune out best solutions.
Model-based auto-tuning predicts the best parameters by platform abstractions but the performance can degrade.
\par
ATLAS uses orthogonal line search to find optimal parameters.
The search takes a long time and the space is not guaranteed to include the optimal points.
In this project, we introduce Capri-based ATLAS search, to reduce the search time and achieve high performance simultaneously.
Capri first takes the results of exhaustive search and trains them offline. During the ATLAS code generation, it provides candidate parameters for ATLAS to run on the platform and pick the best.
\par
We evaluate the system on 12 platforms. Our experiments show that  Capri-based ATLAS search achieves XXX speedup over ATLAS orthogonal line search and it only suffers YYY performance degradation.


\cite{yotov2005search}\cite{sui2016proactive}
 \ldots.

\end{abstract}

%% 2012 ACM Computing Classification System (CSS) concepts
%% Generate at 'http://dl.acm.org/ccs/ccs.cfm'.
\begin{CCSXML}
<ccs2012>
<concept>
<concept_id>10011007.10011006.10011008</concept_id>
<concept_desc>Software and its engineering~General programming languages</concept_desc>
<concept_significance>500</concept_significance>
</concept>
<concept>
<concept_id>10003456.10003457.10003521.10003525</concept_id>
<concept_desc>Social and professional topics~History of programming languages</concept_desc>
<concept_significance>300</concept_significance>
</concept>
</ccs2012>
\end{CCSXML}

\ccsdesc[500]{Software and its engineering~General programming languages}
\ccsdesc[300]{Social and professional topics~History of programming languages}
%% End of generated code


%% Keywords
%% comma separated list
\keywords{keyword1, keyword2, keyword3}  %% \keywords are mandatory in final camera-ready submission


%% \maketitle
%% Note: \maketitle command must come after title commands, author
%% commands, abstract environment, Computing Classification System
%% environment and commands, and keywords command.
\maketitle


\section{Introduction}
\label{sec:intro}


Optimization parameters such as tile sizes and loop unrolling factors are critical to performance and it
is crucial to estimate the optimal value during optimization.
Currently programmers usually rely on optimization techniques provided by traditional compilers, which
is often too generalized to yield desired performance.
General compiler techniques cannot take care of applications such as general matrix multiply(GEMM),
whose performance is heavily based on those optimization parameters.
On the other hand, writing highly optimized code is impractical for average
programmers because it requires huge amount of prior knowledge both for application
itself and the target platform.


%Optimization parameters of high performance programs must be tuned for a given platform.
%how the program is written and how the optimization parameter is set.
\par

Auto-tuning is a straight-forward and one of the simplest ways to find the optimal parameters.
There are two major categories of auto-tuning approaches: exhaustive search and model-based search.
\par
Exhaustive search tries to find out the parameter setting with the best
performance by actively searching every parameter setting in
the design space. All the parameter settings are evaluated and compared by
actual execution.
This technique has been successfully applied to build a variety
of high-performance domain-specific libraries including
dense linear algebra \cite{whaley2001automated, bilmes2014optimizing},
sparse linear algebra \cite{vuduc2005oski}, signal processing
\cite{frigo2005design, puschel2005spiral}, sorting \cite{li2004dynamically},
general stencil operations \cite{kamil2010auto}, etc.

However, exhaustively searching the whole parameter space is slow and usually
impractical for most of high performance computing applications.
The search space is usually too large and the evaluation of parameter settings
with low performance takes much more time ironically.
Consequently, exhuastive search can only be applied to problems with small
optimization space or already pruned spaces.

The second category of auto-tuning is model-based tuning. This kind of methods
usually builds an analytical model to estimates optimal parameters.
The significant advantage of such approach is fast. Good optimization
parameters can be directly derived from the analytical model.
However, it is generally difficult to build a highly accurate analytical model
in practice because of the growing complexity of modern architectures and
applications. The analytical model is too platform dependent and application
specific for practical use.

It is intuitive to come up with the idea that somehow combines the two
approaches together by first pruning the search space with an automatically
generated model, and then evaluating the pruned search space by exhaustive
search. On the other hand, users are not willing
to wait for a day or two, or even an hour to get the most optimized result in
most cases. An ``good enough'' result is adequate. Therefore, approximation can
be introduced to accelerate the optimization process.

There is growing interest in approximate computing as a way of reducing energy
and time required to execute applications. \cite{ansel2011language,
baek2010green, sidiroglou2011managing, swaminathan2015case}. In conventional
computing, programs are usually treated as implementations of mathematical
functions, so there is a precise output for a given input.
However, in many problem domains, it is sufficient to produce an good ``enough''
result: for example, when rendering a image in graphics, it is
acceptable to take computational short-cuts if the graph doesn't hurt user's
experience. In addition, since analytical models are often imprecise for
applications with high complexity. Machine learning models can be a great
substitute. In this paper, we use Capri\cite{sui2016proactive}, a published
approximate software, to build machine learning based models for \gem
performance. We denote parameter settings as knobs or knob settings in
our paper. Capri would take past \atl running data and produce a machine
learning based model. \atl search process would take few top candidate knobs
generated by Capri and provide users with the best knob among the candidates
knobs by evaluating them exhaustively.

The contributions of this papers are:
\begin{itemize}
\item The performance of \gem can be well modeled by machine learning based
models provided by Capri.
\item The proposed Capri-based \atl can speed up the optimization process by 10X.
\item The performance of \gem using the knobs generated by Capri-based \atl
has average of 2\% performance degradation compared to the exhaustive search,
while the original \atl suffers 13\% performance degradation.
\end{itemize}

\par
The rest of this paper is organized as follows. Section \ref{sec:background}
describes the background of ATLAS tuning strategy and Capri approximate
software. Section \ref{sec:design} introduces the design of Capri-based
ATLAS search.
Section \ref{sec:experiment} provides the details of experiment methodology.
Section \ref{sec:evaluation} shows all the experiment results of our system,
analyzing and comparing them with the original ATLAS result.
Section \ref{sec:related} provides an overview of related work.
And finally, section \ref{sec:conclusion} concludes and outlines our possible
future work.

\section{Background}
\label{sec:background}

  \subsection{ATLAS code generator}
  \label{sec:atlas_intro}
  ATLAS uses orthogonal line search to auto-tune its optimization parameters.
  It assumes that the parameters are independent to performance and the optimal values can be
  restricted by some architecture features like number of registers, size of L1 cache.
  Even though ATLAS empirical can generate good parameters, its search space is still not guaranteed to contain
  the optimal point, and there are still many low performance points in the search space which is unnecessary
  to search.

  \subsection{Capri approximate program}
  \label{sec:Capri_intro}
  Capri is a approximate program that helps trade off energy or time with error
  or performance for input programs. Capri focuses on a class of approximate
  programs that we call tunable approximate programs. Intuitively, these
  programs have one or more knobs that can be changed to vary the fidelity of
  produced output. These knobs might control the number of iterations performed
  by a loop \cite{}, determine the precision with which floating-point
  computations are performed \cite{}, and so on.

  There is now a fairly large literature on this subject. Most of these work
  address the solution for \emph{forward problems}: they show that for some
  programes, particular techiques or heuristics can be applied to degrade
  output quality in an acceptable way while reducing energy or running time.
  Another category of approximate problem is called \emph{
  inverse problems}: given some constraint on output quality, how shall we set
  the knobs in order to meet the output quality while reducing the cost (running
  time or error) as much as possible. What makes an approximate problem
  particularly difficult is that for most applications, optimal values of
  knob settings are very dependent on values of inputs. Capri is capable of
  doing both of them as well as the variant introduced by inputs.

  The work flow for Capri can be divided into two phases: training
  phase, testing phase. The training phase is conducted
  offline. Basically, Capri requires the target approximate problem to run
  extensively across the knob space for some inputs and collect cost(running
  time, energy) and error (output quality or perfermance degradation). A set of
  knob-cost and knob-error pairs are formed for each input. For each input, some
  features are extracted in order to represent the input. These sets of
  knob-cost and knob-error pairs with their input features are fed into
  machine learning boxes in order to build a cost model and a error model
  respectively based on input features and different knob settings.

  During the test phase, Capri would take a
  constraint and an input for the target program as inputs, such as ``a knob
  settings which is 90\% as good as the best rendering quality'', the
  controller would consult the error model and find out
  all the knob settings that is predicted as at least 90\% as good as the best
  one, denoted as feasible region of knobs. Then, the Capri consults the
  cost model for each knob in the feasible region and output the knob setting
  with the least cost. This knob setting with be used for this specific input
  that is predicted to produce a 90\% good result with the least cost.

  The offline training phase is heavy but the online testing phase is really
  light, which usually take few seconds to finish. In this project, Capri would
  be used to build the model for \gem performance based on knobs in the code on
  different platforms. \atl would take the advantage of this model to prune
  the optimization exploration space and produce an ``enough good'' \gem
  installation.


  %Auto-tuning is a generally used optimization process, which is also
  %approximateable. Intuitively, instead of replacing auto-tuning completely,
  %the process of auto-tunning can be shortened by pruning the search space.

\section{Motivation}
\label{sec:moti}

\section{Design}
\label{sec:design}

The design of the Capri-based \atl can be separated into four parts: knob
selection, feature selection, the offline training phase and the online testing
phase. Knobs, as descried
The offline training phase captures
the characteristics of the platform features and knob settings by doing
exhaustive runs. Capri uses the profiling data and builds a model for the \gem
performance based on featues and knobs. In the testing phase, the model takes
a platform and a ``top M'' constraint as input, and output the best predicted
knob setting for GEMM for this platform.

  \subsection{Knob selection}
  \label{sec:knobs}

  \subsection{Feature selection}
  \label{sec:features}
  Features are used in Capri as the identification for different inputs.

  \subsection{Offline Training}
  \label{sec:offline_training}

  \subsection{Online Testing}
  \label{sec:online_training}

\begin{figure*}[tbhp]
  \centering
  \begin{subfigure}[b]{1.0\linewidth}
    \centering
    \includegraphics[width=0.9\textwidth]{images/offline_training.png}
    \caption{Offline training phase: \atl profiling and Capri model building}
  \end{subfigure}
  \begin{subfigure}[b]{1.0\linewidth}
    \centering
    \includegraphics[width=0.9\textwidth]{images/testing.png}
    \caption{Online testing phase: Capri model searching and \atl execution}
  \end{subfigure}
  \caption{Capri-based \atl design diagram}
  \label{fig:design}
\end{figure*}

$mflops = f(platform, NB, MU, NU, KU)$

\section{Experiment Methodology}
\label{sec:experiment}

In the experiment, we only consider single precision, real number matrix multiplication. We implement 
an exhaustive search on ATLAS. The space of each parameter in $NB$, $MU$, $NU$, $KU$ are not smaller 
than the ATLAS orthogonal search, and the exhaustive search tries every possible combination of them. 
On a typical platform as we mentioned in Section \ref{atlas_intro}, it takes more than 10 hours to finish 
the search on 12960 points. 



\section{Evaluation}
\label{sec:evaluation}

  \subsection{Exhaustive search vs \atl orthogonal search}
  \label{sec:exhaustiveVSorthogonal}

  \begin{figure*}[tbhp]
    \centering
    \includegraphics[width=0.9\textwidth]{images/exhaustiveVsorthogonal.png}
    \caption{Searching time comparison between exhaustive search and \atl orthogonal search}
    \label{fig:exhaustiveVsorthogonal}
  \end{figure*}

  \subsection{GEMM performance model}
  \label{sec:GEMMperf}

  \subsection{Capri-based ATLAS searching time}
  \label{sec:capri_atlas_searching}

  \begin{figure*}[tbhp]
    \centering
    \includegraphics[width=0.9\textwidth]{images/timespeedup.png}
    \caption{Capri-based ATLAS searching time speedup}
    \label{fig:platforms}
  \end{figure*}

  \subsection{Capri-based ATLAS performance}
  \label{sec:capri_atlas_performance}

  \begin{figure*}[tbhp]
    \centering
    \begin{subfigure}[b]{1.0\linewidth}
      \centering
      \includegraphics[width=0.9\textwidth]{images/all_perf.png}
      \caption{Training set contains all the platforms except the testing platform}
      \label{fig:all_perf}
    \end{subfigure}
    \begin{subfigure}[b]{1.0\linewidth}
      \centering
      \includegraphics[width=0.9\textwidth]{images/corei_perf.png}
      \caption{Training set contains all the ``corei'' architecture platforms except the testing platform}
      \label{fig:corei_perf}
    \end{subfigure}
    \begin{subfigure}[b]{1.0\linewidth}
      \centering
      \includegraphics[width=0.9\textwidth]{images/specific_perf.png}
      \caption{Training set contains all the specific platforms except the testing platform}
      \label{fig:specific_perf}
    \end{subfigure}
  \caption{Capri-based \atl design diagram}
  \end{figure*}

  \subsection{Parameter sensitivity}
  \label{sec:parametersensitivity}

    \subsubsection{Training set}
    \label{sec:training_set}

    \subsubsection{Top M constraint}
    \label{sec:top_m}

\section{Related Work}
\label{sec:related}

There are many heuristics on pruning the searching space and improving performance of auto-tuning.

ATLAS\cite{yotov2005search} uses orthogonal to reduce the search space. It assumes that 
the optimization parameters are independent to each other and they are restricted by some hardware features.
However as we have seen in the experiment, the assumption is not always true and its generated code
may have low performance. A similar technique \cite{li2009note} is later implemented on GPU to keep up
with rapid change in hardware. FFTW\cite{frigo2005design} adopts dynamic programming where program of large size problem
is constructed from previously generated code for small size. The inherent assumption of dynamic programming is that,
the best code of a transform is independent of the calling context. This assumption holds for the arithmetic cost (which implies
that DP produces the optimal solution), but not for the runtime of transform algorithms.
SPIRAL\cite{puschel2005spiral} not only adopts dynamic programs,
but also tries exhaustive search, random search, evolutionary search and hill climbing heuristic. The property 
of evolutionary search and hill climbing makes the generated solution may be local optimal. \par

Model-driven auto-tuning is another approach. ATLAS\cite{yotov2005search} adops a model from a heuristic based on
hardware parameters. SPIRAL\cite{puschel2005spiral} build and learn a model from the features a nodes in the divide-and-conquer tree.
The feature must be carefully selected and the built model shows 10\% to 20\% performance loss. \par

It requires expertise on both hardware architectures and applications to build an accurate model. Therefore a lot of 
effort has been spent in using machine-learning to automatically learn a model. Li et al.\cite{li2004dynamically} apply machine learning 
technique to build a model for sorting. The machine learning technique predicts the best algorithm as a function of the 
characteristics of the input data set and the performance of the difference algorithms on the target platform. Falch at al.\cite{falch2015machine}
focuses on OpenCL which targets on heterogeneous system and often suffers from performance portability. They run on a random
subset of the entire tuning parameter space, and the results are used to build an artificial neural network based model. 
The model can be used to find interesting parts of the parameter space for further search. Yigitbasi et al.\cite{yigitbasi2013towards} focus on MapReduce.
They selected support vector regression model(SVR) as the most accurate machine learning model among several candidates, by 
training them with diverse MapReduce applications and cluster configurations. Bergstra et al.\cite{bergstra2012machine} build a machine learning model
with boosted regression tree and generate high performance library on multicore GPU platforms.





\section{Conclusion}
\label{sec:conclusion}


In this project, we adopts Capri to the search procedure of ATLAS. 
Capri is able to build a model to reduce the search time and predict high performance optimization parameters simultaneously.
Capri first takes the results of exhaustive search on many platforms and trains them offline. 
During the ATLAS code generation, it provides candidate parameters for ATLAS to run on the actual platform and pick the best.
\par
We evaluate the system on 12 platforms with 2 ARM and 10 Intel platforms. Our experiments show that over ATLAS orthogonal line search, Capri-based ATLAS search achieves an average of 10.9X speedup and the MFLOPS performance of generated code is 11\% higher. It only suffers 2\% performance degradation from best of exhaustive search.



%% Acknowledgments
\begin{acks}                            %% acks environment is optional
                                        %% contents suppressed with 'anonymous'
  %% extraction tools.
  This work is based on ATLAS software and Capri project. The authors would
  like to thank Swarnendu(UTCS), Xin Sui(Tableau Software) and all the ATLAS
  developers.

  Yan Pei: I really enjoy this class. The course proves again that reading
  paper with discussion is be a great method of learning. I learned a lot
  during this class. The abstractions and high level ideas introduced in
  this class should be useful in my research career. On the other hand, I
  really wanna express my gratitude to Dr. Pingali and all my classmates.
  Thank you for preparing classes and presentations for us.

  Jiayuan He:

\end{acks}


%% Bibliography
%\bibliography{reference}


%% Appendix
\appendix
\section{Appendix}

Text of appendix \ldots

\end{document}
