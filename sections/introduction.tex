\section{Introduction}
\label{sec:intro}

A key step in program optimization is the estimation of optimal values for
parameters such as tile sizes and loop unrolling factors. Currently programmers
usually rely on optimization techiques provided by traditional compilers, which
is often too generalized to produce desired performance. General compilers
cannot take care of operations such as general matrix multiply(GEMM), whose
performance is heavily based on the how the programme is written and how the
parameter is set.
\par
=============There needs something=============
Optimization parameters of high performance programs must be tuned for a given platform. 
Auto-tuning is a straight-forward and one of the simplest way to find the optimal parameters.
Empirical search and model-based tuning are two major kinds of approach. 
Empirical search tries every data point in the parameter space and evaluate the performance of
them to get the optimal solution. 
However, exaustively searching the whole parameter space is slow and impractical. 
The search space is usually too large and contains lots of low performance points where
it is unneccesary to waste time. 
\par
Programmers with hardware and application knowledge can prune the space to reduce the search time.
ATLAS uses orthogonal line search to auto-tune its optimization parameters. 
It assumes that the parameters are independent to performance and the optimizal values can be 
restricted by some architecture features like number of registers, size of L1 cache. 




