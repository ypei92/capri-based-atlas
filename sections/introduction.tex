\section{Introduction}
\label{sec:intro}

A key step in program optimization is the estimation of optimal values for
parameters such as tile sizes and loop unrolling factors. Currently programmers
usually rely on optimization techiques provided by traditional compilers, which
is often too generalized to produce desired performance. General compilers
cannot take care of operations such as general matrix multiply(GEMM), whose
performance is heavily based on the how the programme is written and how the
parameter is set.
\par
=============There needs something=============
Optimization parameters of high performance programs must be tuned for a given platform. 
Auto-tuning is a straight-forward and one of the simplest way to find the optimal parameters.
There are two major kinds of auto-tuning approach: empirical search and model-based tuning. 
\par
Empirical search tries to find the best performing code by searching every data point in 
the parameter space and evaluate the performance of them on the actual hardware.
It has been successfully applied to build a variety
of high-performance domain-specific libraries including
dense linear algebra [Clint Whaley et al., 2001, Bilmes et al.,
1997], sparse linear algebra [Vuduc et al., 2005], signal processing
[Frigo and Johnson, 2005], sorting [Li et al., 2004],
general stencil operations [Kamil et al., 2010], etc.
\par
However, exaustively searching the whole parameter space is slow and impractical. 
The search space is usually too large for modern machines and contains lots of low performance points where
searching is unnecessary. Consequently, most proposed searching methods
prune the space with hard-coded heuristics and may lose generality on new platforms.
\par
ATLAS uses orthogonal line search to auto-tune its optimization parameters. 
It assumes that the parameters are independent to performance and the optimal values can be 
restricted by some architecture features like number of registers, size of L1 cache. 
Even though ATLAS empirical can generate good parameters, its search space is still not guaranteed to contain 
the optimal point, and there are still many low performance points in the search space which is unnecessary 
to search.
\par
The second approach of auto-tuning is model-based tuning, which builds an analytical model
to estimates optimal parameters. The significant advantage of such approach is fast. Good optimization
parameters can be directly derived from the model. 
However in practice, it is generally difficult to build a highly accurate model bacause of the growing complexity
of modern architectures and applications. The analytical model is inflexible and has to be built by expert of both architecture
and application.

\par
To address this weakness, it is intuitive to combine the two approaches together by first pruning the search
space with an automatically generated analytical model, and then evaluate candidate data points generated by the model
to get the best parameters.
\par
=====================an empty line==========
\par
CAPRI HERE

The contributions of this papers are:
1.
2.
3.

\par
The rest of this paper is organized as follows. Section [motivation and background] describes the details of ATLAS
orthogonal line search and Capri system.
Section [Design] introduces the implementation of Capri-based ATLAS search.
Section [Experiment Methodology] provides the details of experiment methodology.
Section [results] shows all the results of our system, analyze and compare them with ATLAS orthogonal search. 
Section [related work] provides an overview of related work.
And finally, section [conclusion] concldes and outlines our possible future work.








